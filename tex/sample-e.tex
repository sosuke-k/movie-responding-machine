\documentclass{deime}
%\usepackage[dvipdfm]{graphicx}
%\usepackage{latexsym}
%\usepackage{txfonts}
%\usepackage[fleqn]{amsmath}
%\usepackage[psamsfonts]{amssymb}
%\usepackage[deluxe]{otf}

% 印刷位置調整 %
% 必要に応じて値を変更してください.
\hoffset -10mm % <-- 左に 10mm 移動
\voffset -10mm % <-- 上に 10mm 移動

\newcommand{\AmSLaTeX}{%
 $\mathcal A$\lower.4ex\hbox{$\!\mathcal M\!$}$\mathcal S$-\LaTeX}
\newcommand{\PS}{{\scshape Post\-Script}}
\def\BibTeX{{\rmfamily B\kern-.05em{\scshape i\kern-.025em b}\kern-.08em
 T\kern-.1667em\lower.7ex\hbox{E}\kern-.125em X}}

\papernumber{DEIM Forum 2016 XX-Y}

\etitle{DEIM Forum 2016 Class File }
%\esubtitle{Subtitle} <- uncomment if you need subtitle.
\authorlist{%
 \authorentry[nishinosono@n-univ.ac.jp]{西之園 萌絵}{Moe Nishinosono}{UnivN}% 
 \authorentry[saikawa@n-univ.ac.jp]{犀川 創平}{Sohei Saikawa}{UnivN}% 
 \authorentry[magata@mlab.co.jp]{真賀田 四季}{Shiki Magata}{Mlab}% 
}
\affiliate[UnivN]{N大学工学部建築学科\hskip1zw
  〒464--8603 愛知県名古屋市千種区不老町}
 {School of Engineering, N University\\
  1--1--1 Furo-cho, Chikusa-ku, Nagoya, 410--2415 Japan
  410--2415 Japan}
\affiliate[Mlab]{真賀田研究所\hskip1zw
  〒100--0001 東京都千代田区千代田1番1号}
 {Magata Laboratory,\\
  1--1 Chiyoda, Chiyoda, Tokyo,
  100--0001 Japan}

%\MailAddress{$\dagger$hanako@deim.ac.jp,
% $\dagger\dagger$\{taro,jiro\}@jforum.co.jp}

\begin{document}
\pagestyle{empty}
\begin{eabstract}
Paper format for DEIM Forum 2016 Proceedings.
\end{eabstract}
\begin{ekeyword}
p\LaTeXe\ class file, typesetting
\end{ekeyword}
\maketitle

\section{タイトル・概要に関して}

1ページ目上部には,タイトル,発表者氏名,所属,住所,メールアドレス,キー
ワードの和文と英文及びあらまし(about 100 words)を,それぞ
れ記述してください.なお、和文論文については英文タイトル,アブストラク
ト等は削除して頂いて構いません。下記のコマンドで講演番号を挿入して下さ
い.
\begin{verbatim}
 \papernumber{DEIM Forum 2016 XX-Y}
\end{verbatim}
XXはセッション番号(例:1A, 3B),Yはセッション内での発表順(1, 2, ...)です.
番号についてはプログラムをご覧ください.
なお,プログラム決定前の初回投稿時にはXX-Yの部分の記入は不要です.

\section{原稿提出枚数}

所定のページ数(4~8ページ)を厳守してください.
Ph.Dセッション投稿者は8ページを推奨します.

\section{原稿の書き方}

原稿のスタイルは,A4サイズで,9ポイントのフォントを使用し,2段組み,シングルスペースとして下さい.

\vspace{5em}

\begin{thebibliography}{99}
\bibitem{Codd1970}
  E. F. Codd, 
  ``A Relational Model of Data for Large Shared Data Banks,''
  Communications of the {ACM} (CACM), Vol. 13, No. 6, pp. 377--387, 1970.
\end{thebibliography}


\end{document}
